\documentclass{article}

%%%%%%%%%%%%%%%%%%%%%%%%%%%%%%%%%%%%%%%%%%%%%%%%%%%%%%%%%%%%%%%%%%%%%%%%%%%%%%%%

\usepackage{amsmath,amssymb,amsthm}

%%%%%%%%%%%%%%%%%%%%%%%%%%%%%%%%%%%%%%%%

\theoremstyle{definition}
\newtheorem{invariant}{Invariant}
\newtheorem{defn}{Definition}
\newtheorem{theorem}{Theorem}
\newtheorem{lemma}{Lemma}
\newtheorem{example}{Example}

\newcommand{\p}{\ensuremath p}
\newcommand{\q}{\ensuremath q}

\newcommand{\set}{\mathcal}
\newcommand{\radius}{r}
\newcommand{\dispersion}{d}
\newcommand{\distf}{d}
\newcommand{\dpdim}{\text{dpdim}}
\newcommand{\dpnum}{\text{dpnum}}

\newcommand{\dist}[2]{\distf({#1},{#2})}
\newcommand{\level}{\text{\ttfamily level}}
\newcommand{\children}{\text{\ttfamily children}}
\newcommand{\covdist}{\text{\ttfamily covdist}}

%%%%%%%%%%%%%%%%%%%%%%%%%%%%%%%%%%%%%%%%%%%%%%%%%%%%%%%%%%%%%%%%%%%%%%%%%%%%%%%%

\title{Cover Trees}
\author{Mike Izbicki}

\begin{document}
\maketitle

\section{Introduction}

\begin{defn}
A metric space is a set $\set X$ and a distance function $\distf : \set X \times \set X \to \mathbb R$.
\end{defn}

\section{Implementation}
A cover tree satisfies the following properties.

\begin{invariant}
Every node $\p$ has an associated integer $\level(\p)$.
For all nodes $\q\in\children(\p)$, $\level(\q) < \level(\p)$.
\end{invariant}

\begin{invariant}
Every node $\p$ has an associated real number $\covdist(\p)=2^{\level(\p)}$.
For all nodes $\q\in\children(\p)$, $\dist \p \q \le \covdist(\p)$.
\end{invariant}

\begin{invariant}
For all nodes $\q_1,\q_2\in\children(\p)$, $\dist {\q_1} {\q_2} \le \covdist(\p)$.
\end{invariant}

\section{Analysis}

\begin{defn}
%A ball of radius $\delta$ in a metric space $\set X$ is defined to be
A ball is defined as
\begin{equation}
B_\set X(x,\delta) = \{ y : y\in\set X, \dist{x}{y} \le \delta \}.
\end{equation}
\end{defn}

\begin{defn}
A $\delta$-packing of a set $\set X$ with respect to a distance $\distf$ is a set $\{x_1,x_2,...,x_M\} \subseteq \set X$ such that $\dist{x_i}{x_j} > \delta$ for all distinct $i,j\in[M]$.
The $\delta$-packing number $M_\delta (\set X)$ is the cardinality of the largest $\delta$-packing.
\end{defn}

\begin{defn}
The double-packing number of a set $\set X$ is defined as
\begin{equation}
\dpnum(\set X) = \max_{x\in\set X,\delta\in\mathbb R^+} M_\delta( B(x,2\delta))
.
\end{equation}
The double-packing dimension is defined to be the base 2 logarithm of the double-packing number.
That is,
\begin{equation}
\dpdim(\set X) = \lg \dpnum(\set X)
.
\end{equation}
\end{defn}

\begin{lemma}
Let $\set X_1$ and $\set X_2$ be two sets satisfying $\set X_1 \subseteq \set X_2$,
and let $\distf$ be a metric over both sets.
Then $\dpnum(\set X_1) \le \dpnum(\set X_2)$.
\end{lemma}
\begin{proof}
Let $x$ be a point in $\set X_1$ and $\delta\in\mathbb R^+$.
Then any valid $\delta$-packing of $B_{\set X_1}(x,2\delta)$ is also a valid $\delta$-packing of $B_{\set X_2}(x,2\delta)$.
\end{proof}

\begin{lemma}
Let $\set X_1 \subset \set X_2$,
and $x$ be a point in $\set X_2$ but not in $\set X_1$.
Then,
\begin{equation}
\dpnum(\set X_1 \cup\{x\}) \le \dpnum (\set X_1) + 1
.
\end{equation}
\end{lemma}
\begin{proof}
%Let $p$ be a maximal $\delta$-packing for $\set X\cup \{x\}$.
%If $x\not\in p$, then $p$ is also a maximal $\delta$-packing for $\set X$,
%and hence $\dpnum(\set X\cup\{x\}) = \dpnum(\set X)$.
Let $p$ be a maximal $\delta$-packing of $\set X$.
Assume for contradiction that there exists a $\delta$-packing $p'$ of $\set X\cup\{x\}$ such that $|p'| > |p| + 1$.
If $x\not\in p'$, then $p'$ is a $\delta$-packing of $\set X$.
But $|p'| > |p|$, which violates the assumption that $p$ is maximal.
If $x\in p$, then the set $p'-\{x\}$ is a packing of $\set X$.
\end{proof}

\begin{lemma}
Let $\set X_1$ and $\set X_2$ be metric spaces with the same distance function $\distf$.
Then,
\begin{equation}
\dpdim(\set X_1 \cup \set X_2) \le \dpdim(\set X_1) + \dpdim(\set X_2)
.
\end{equation}
\end{lemma}

\begin{defn}
The radius of a dataset is defined as
\begin{equation}
\radius(\set X) = {\displaystyle\max_{x_1,x_2\in \set X} \dist{x_1}{x_2}}
%\radius(\set X) = \max\{\dist{x_1}{x_2} : x_1\in \set X, x_2\in \set X\}
,
\end{equation}
the dispersion of a dataset is defined as
\begin{equation}
\dispersion(\set X) = {\displaystyle\min_{x_1,x_2\in \set X : x_1\neq x_2} \dist{x_1}{x_2}}
,
\end{equation}
and the condition number of a dataset is their ratio
\begin{equation}
\kappa(\set X)
= \frac
    {\radius(\set X)}
    {\dispersion(\set X)}
    %{\displaystyle\max_{x_1,x_2\in \set X} \dist{x_1}{x_2}}
    %{\displaystyle\min_{x_1,x_2\in \set X} \dist{x_1}{x_2}}
    .
\end{equation}
\end{defn}
\begin{lemma}
The depth of a cover tree is bounded by the log of the condition number.
\end{lemma}

\begin{defn}
The doubling dimension of a metric space $(\set X,\distf)$
\begin{equation}
c = \lg \max_{x\in\set X} \frac{\mu B(x,2\delta)}{\mu B(x,\delta)}
\end{equation}
\end{defn}

\begin{theorem}
Insertion takes time $O(c^{12} \log n)$.
\end{theorem}

\begin{example}
Consider a data set of $m$ points in $\mathbb{R}$.
Let $x_0=1$, and $x_t = x_{t-1}/2$.
Then there is a valid cover tree over this data set with height $m$.
\end{example}

\end{document}
